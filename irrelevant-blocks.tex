\subsection{Previous modeling approaches and calibration attempts}
\label{sec:model-setup}

A previous bi-dimensional hydro-morphodynamic model of the fishway was developed by\citeA{scolari2025hydromorphodynamic}. The study covered the same area as in the present study and simulates the hydrodynamics and morphodynamics with TELEMAC and GAIA to detect the risk of clogging in the vicinity of two large wood logs by examining variations in the fine sediment fraction (FSF) on the surface and subsurface of the riverbed after an artificial flushing event of 9.7~m\textsuperscript{3}~s\textsuperscript{-1}.
The approach considered the riverbed heterogeneity along the channel and defined eleven roughness zones which included floodplains and riverbed. The study used a manually (deterministic) calibrated hydrodynamic model version, in which flow velocities and water depths were used as calibration targets. The Nikuradse roughness heights of eleven roughness zones along with the Shields parameter for each grain class were used as calibration parameters. The calibration procedure considered a single Nikuradse roughness height for the entire riverbed, including the wood logs regions. 



%For $MC$ Monte Carlo samples (or realizations), the matricial representation of the metamodel outputs for velocity $\mathbf{V}$ and water depth $\mathbf{H}$ or a combination of both variables $\mathbf{VH}$ and at P measurement locations is:
%
%\begin{equation} \label{eq:}
%\mathbf{V} = \begin{pmatrix}
%	v_{11} & v_{12} & \dots & v_{1P} \\
%	v_{21} & v_{22} & \dots & v_{2P} \\
%	\vdots & \vdots & \ddots & \vdots \\
%	v_{MC1} & v_{MC2} & \dots & v_{MCP}
%\end{pmatrix}
%\end{equation}
%
%\begin{equation} \label{eq:}
%\mathbf{H} = \begin{pmatrix}
%	h_{11} & h_{12} & \dots & h_{1P} \\
%	h_{21} & h_{22} & \dots & h_{2P} \\
%	\vdots & \vdots & \ddots & \vdots \\
%	h_{MC1} & h_{MC2} & \dots & h_{MCP}
%\end{pmatrix}
%\end{equation}
%
%\begin{equation} \label{eq:}
%\mathbf{VH} = \begin{pmatrix}
%	v_{11} & h_{11} & v_{12} & h_{12} & \dots & v_{1P} & h_{1P} \\
%	v_{21} & h_{21} & v_{22} & h_{22} & \dots & v_{2P} & h_{2P} \\
%	\vdots & \vdots & \vdots & \vdots & \ddots & \vdots & \vdots \\
%	v_{MC1} & h_{MC1} & v_{MC2} & h_{MC2} & \dots & v_{MCP} & h_{MCP}
%\end{pmatrix}
%\end{equation}
%
%
